\PassOptionsToPackage{unicode=true}{hyperref} % options for packages loaded elsewhere
\PassOptionsToPackage{hyphens}{url}
%
\documentclass[]{book}
\usepackage{lmodern}
\usepackage{amssymb,amsmath}
\usepackage{ifxetex,ifluatex}
\usepackage{fixltx2e} % provides \textsubscript
\ifnum 0\ifxetex 1\fi\ifluatex 1\fi=0 % if pdftex
  \usepackage[T1]{fontenc}
  \usepackage[utf8]{inputenc}
  \usepackage{textcomp} % provides euro and other symbols
\else % if luatex or xelatex
  \usepackage{unicode-math}
  \defaultfontfeatures{Ligatures=TeX,Scale=MatchLowercase}
\fi
% use upquote if available, for straight quotes in verbatim environments
\IfFileExists{upquote.sty}{\usepackage{upquote}}{}
% use microtype if available
\IfFileExists{microtype.sty}{%
\usepackage[]{microtype}
\UseMicrotypeSet[protrusion]{basicmath} % disable protrusion for tt fonts
}{}
\IfFileExists{parskip.sty}{%
\usepackage{parskip}
}{% else
\setlength{\parindent}{0pt}
\setlength{\parskip}{6pt plus 2pt minus 1pt}
}
\usepackage{hyperref}
\hypersetup{
            pdftitle={Foundations of data science},
            pdfauthor={Steffen Mæland, PierGianLuca Porta Mana},
            pdfborder={0 0 0},
            breaklinks=true}
\urlstyle{same}  % don't use monospace font for urls
\usepackage{longtable,booktabs}
% Fix footnotes in tables (requires footnote package)
\IfFileExists{footnote.sty}{\usepackage{footnote}\makesavenoteenv{longtable}}{}
\usepackage{graphicx,grffile}
\makeatletter
\def\maxwidth{\ifdim\Gin@nat@width>\linewidth\linewidth\else\Gin@nat@width\fi}
\def\maxheight{\ifdim\Gin@nat@height>\textheight\textheight\else\Gin@nat@height\fi}
\makeatother
% Scale images if necessary, so that they will not overflow the page
% margins by default, and it is still possible to overwrite the defaults
% using explicit options in \includegraphics[width, height, ...]{}
\setkeys{Gin}{width=\maxwidth,height=\maxheight,keepaspectratio}
\setlength{\emergencystretch}{3em}  % prevent overfull lines
\providecommand{\tightlist}{%
  \setlength{\itemsep}{0pt}\setlength{\parskip}{0pt}}
\setcounter{secnumdepth}{5}
% Redefines (sub)paragraphs to behave more like sections
\ifx\paragraph\undefined\else
\let\oldparagraph\paragraph
\renewcommand{\paragraph}[1]{\oldparagraph{#1}\mbox{}}
\fi
\ifx\subparagraph\undefined\else
\let\oldsubparagraph\subparagraph
\renewcommand{\subparagraph}[1]{\oldsubparagraph{#1}\mbox{}}
\fi

% set default figure placement to htbp
\makeatletter
\def\fps@figure{htbp}
\makeatother

\usepackage{booktabs}
\usepackage{amsthm}
\makeatletter
\def\thm@space@setup{%
  \thm@preskip=8pt plus 2pt minus 4pt
  \thm@postskip=\thm@preskip
}
\makeatother
\usepackage[]{natbib}
\bibliographystyle{apalike}

\title{Foundations of data science}
\author{Steffen Mæland, PierGianLuca Porta Mana}
\date{2023-05-04}

\begin{document}
\maketitle

{
\setcounter{tocdepth}{1}
\tableofcontents
}
\hypertarget{preface}{%
\chapter*{Preface}\label{preface}}
\addcontentsline{toc}{chapter}{Preface}

Under construction

\hypertarget{intro}{%
\chapter{Introduction}\label{intro}}

To be written: motivation and structure of this course.

\hypertarget{truth-inference-and-probability-inference}{%
\chapter{Truth inference and probability inference}\label{truth-inference-and-probability-inference}}

\hypertarget{sentences}{%
\section{Sentences}\label{sentences}}

Facts, hypotheses, questions, decisions -- and data are communicated through language and sentences. You may say ``well, data can be just numbers, they don't need to be communicated through sentences''. But is that true?

I give you this number: ``5''. OK it's a number, but what's it about? what should you do with it? is that ``data''? Instead, if I tell you: ``The number of lectures in this course is 5'' then I have given you a piece of information, a datum (even if it is actually false). Underlying any piece of information, hypothesis, or datum, there is always a \emph{sentence} -- also called \emph{statement} or \emph{proposition}\footnote{These terms are not equivalent in Logic, but we'll use them as synonyms here.} -- that gives you the meaning and context of that datum.

\hypertarget{well-posed-and-ill-posed-statements}{%
\section{Well-posed and ill-posed statements}\label{well-posed-and-ill-posed-statements}}

We face problems when the sentences that should convey information are not clear. Suppose that an electric-car model \href{https://ev-database.org/cheatsheet/energy-consumption-electric-car}{consumes 150~Wh/km} and \href{https://ev-database.org/cheatsheet/range-electric-car}{has a range of 200~km}; a second car model consumes 250~Wh/km and has a range of 600~km. Someone asks you: ``which model is better?''. Well, it isn't clear how you should answer; what does ``better'' mean? If it refers to consumption, then the first car is ``better''. If it refers to range, then the second car is ``better''. If it refers to a combination of these two characteristics, or to something else, then you simply can't answer. Here we have a problem with querying and giving data, because the statement underlying such query is not clear. We say that statement is not \textbf{well-posed}, or that it is \textbf{ill-posed}.

This may seem an obvious discussion to you. Yet you'd be surprised by how often unclear statements appear in scientific papers about data engineering! Not seldom we find discussions and disagreements that actually come from unclear underlying statements, that two parties interpret in different ways.

As a data engineer, you'll often have the upper hand if you are on the lookout for ill-posed statements. Whenever you face an important question, or you're given an important piece of information, or you must provide an important piece of information, always take a little time to examine whether the question or information is actually well-posed.

\begin{itemize}
\tightlist
\item
  \emph{Exercise: give actual paper to analyse}
\end{itemize}

\hypertarget{truth-falsity-and-their-consistency}{%
\section{Truth, falsity, and their consistency}\label{truth-falsity-and-their-consistency}}

\hypertarget{inferences-without-uncertainty-the-truth-calculus}{%
\section{Inferences without uncertainty: the truth calculus}\label{inferences-without-uncertainty-the-truth-calculus}}

\hypertarget{making-room-for-uncertainty-plausibility-credibility-degree-of-belief-probability}{%
\section{\texorpdfstring{Making room for uncertainty:Plausibility, credibility, degree of belief, probability}{Making room for uncertainty: Plausibility, credibility, degree of belief, probability}}\label{making-room-for-uncertainty-plausibility-credibility-degree-of-belief-probability}}

\hypertarget{inferences-with-uncertainty-the-probability-calculus}{%
\section{Inferences with uncertainty: the probability calculus}\label{inferences-with-uncertainty-the-probability-calculus}}

\hypertarget{the-three-fundamental-laws-of-inference}{%
\subsection{The Three Fundamental Laws of inference}\label{the-three-fundamental-laws-of-inference}}

\begin{itemize}
\item
  \emph{Exercise: \href{The_Monty_Hall_problem-exercise.pdf}{Monty-Hall problem \& variations}}
\item
  \emph{Exercise: clinical test \& diagnosis}
\end{itemize}

\hypertarget{bayess-theorem}{%
\subsection{Bayes's theorem}\label{bayess-theorem}}

\hypertarget{common-points-of-certain-and-uncertain-inference}{%
\section{Common points of certain and uncertain inference}\label{common-points-of-certain-and-uncertain-inference}}

\begin{quote}
\emph{No premises? No conclusions!}
\end{quote}

\hypertarget{data-and-information}{%
\chapter{Data and information}\label{data-and-information}}

\hypertarget{kinds-of-data}{%
\section{Kinds of data}\label{kinds-of-data}}

\hypertarget{binary}{%
\subsection{Binary}\label{binary}}

\hypertarget{nominal}{%
\subsection{Nominal}\label{nominal}}

\hypertarget{ordinal}{%
\subsection{Ordinal}\label{ordinal}}

\hypertarget{continuous}{%
\subsection{Continuous}\label{continuous}}

\begin{itemize}
\item
  unbounded
\item
  bounded
\item
  censored
\end{itemize}

\hypertarget{complex-data}{%
\subsection{Complex data}\label{complex-data}}

2D, 3D, images, graphs, etc.

\hypertarget{soft-data}{%
\subsection{``Soft'' data}\label{soft-data}}

\begin{itemize}
\item
  orders of magnitude
\item
  physical bounds
\end{itemize}

\hypertarget{data-transformations}{%
\section{Data transformations}\label{data-transformations}}

\begin{itemize}
\item
  log
\item
  probit
\item
  logit
\end{itemize}

\hypertarget{allocation-of-uncertainty-among-possible-data-values-probability-distributions}{%
\chapter{Allocation of uncertainty among possible data values: probability distributions}\label{allocation-of-uncertainty-among-possible-data-values-probability-distributions}}

\hypertarget{the-difference-between-statistics-and-probability-theory}{%
\section{The difference between Statistics and Probability Theory}\label{the-difference-between-statistics-and-probability-theory}}

\emph{Statistics} is the study of collective properties of collections of data. It does not imply that there is any uncertainty.

\emph{Probability theory} is the quantification and propagation of uncertainty. It does not imply that we have collections of data.

\hypertarget{whats-distributed}{%
\section{What's ``distributed''?}\label{whats-distributed}}

Difference between distribution of probability and distribution of (a collection of) data.

\hypertarget{distributions-of-probability}{%
\section{Distributions of probability}\label{distributions-of-probability}}

\hypertarget{representations}{%
\subsection{Representations}\label{representations}}

\begin{itemize}
\item
  Density function
\item
  Histogram
\item
  Scatter plot
\end{itemize}

Behaviour of representations under transformations of data.

\hypertarget{summaries-of-distributions-of-probability}{%
\section{Summaries of distributions of probability}\label{summaries-of-distributions-of-probability}}

\hypertarget{location}{%
\subsection{Location}\label{location}}

Median, mean

\hypertarget{dispersion-or-range}{%
\subsection{Dispersion or range}\label{dispersion-or-range}}

Quantiles \& quartiles, interquartile range, median absolute deviation, standard deviation, half-range

\hypertarget{resolution}{%
\subsection{Resolution}\label{resolution}}

Differential entropy

\hypertarget{behaviour-of-summaries-under-transformations-of-data-and-errors-in-data}{%
\subsection{Behaviour of summaries under transformations of data and errors in data}\label{behaviour-of-summaries-under-transformations-of-data-and-errors-in-data}}

\hypertarget{outliers-and-out-of-population-data}{%
\section{Outliers and out-of-population data}\label{outliers-and-out-of-population-data}}

(Warnings against tail-cutting and similar nonsense-practices)

\hypertarget{marginal-and-conditional-distributions-of-probability}{%
\section{Marginal and conditional distributions of probability}\label{marginal-and-conditional-distributions-of-probability}}

\hypertarget{collecting-and-sampling-data}{%
\section{Collecting and sampling data}\label{collecting-and-sampling-data}}

\hypertarget{representative-samples}{%
\subsection{``Representative'' samples}\label{representative-samples}}

Size of minimal representative sample = (2\^{}entropy)/precision

\begin{itemize}
\tightlist
\item
  \emph{Exercise: data with 14 binary variates, 10000 samples}
\end{itemize}

\hypertarget{unavoidable-sampling-biases}{%
\subsection{Unavoidable sampling biases}\label{unavoidable-sampling-biases}}

In high dimensions, all datasets are outliers.

Data splits and cross-validation cannot correct sampling biases

\hypertarget{quirks-and-warnings-about-high-dimensional-data}{%
\section{Quirks and warnings about high-dimensional data}\label{quirks-and-warnings-about-high-dimensional-data}}

\hypertarget{making-decisions}{%
\chapter{Making decisions}\label{making-decisions}}

\hypertarget{decisions-possible-situations-and-consequences}{%
\section{Decisions, possible situations, and consequences}\label{decisions-possible-situations-and-consequences}}

\hypertarget{gains-and-losses-utilities}{%
\section{Gains and losses: utilities}\label{gains-and-losses-utilities}}

\hypertarget{factors-that-enter-utility-quantification}{%
\subsection{Factors that enter utility quantification}\label{factors-that-enter-utility-quantification}}

Utilities can rarely be assigned a priori.

\hypertarget{making-decisions-under-uncertainty-maximization-of-expected-utility}{%
\section{Making decisions under uncertainty: maximization of expected utility}\label{making-decisions-under-uncertainty-maximization-of-expected-utility}}

\hypertarget{the-most-general-inference-problem}{%
\chapter{The most general inference problem}\label{the-most-general-inference-problem}}

\hypertarget{literature}{%
\chapter{Literature}\label{literature}}

Here is a review of existing methods.

\bibliography{portamanabib.bib,book.bib,packages.bib}

\end{document}
